% !TEX encoding = UTF-8 Unicode
\documentclass[a4paper]{article}

\usepackage{color}
\usepackage{url}
\usepackage[T2A]{fontenc} % enable Cyrillic fonts
\usepackage[utf8]{inputenc} % make weird characters work
\usepackage{graphicx}

\usepackage[english,serbian]{babel}
%\usepackage[english,serbianc]{babel} %ukljuciti babel sa ovim opcijama, umesto gornjim, ukoliko se koristi cirilica

\usepackage[unicode]{hyperref}
\hypersetup{colorlinks,citecolor=green,filecolor=green,linkcolor=blue,urlcolor=blue}

\usepackage{listings}

%\newtheorem{primer}{Пример}[section] %ćirilični primer
\newtheorem{primer}{Primer}[section]

\definecolor{mygreen}{rgb}{0,0.6,0}
\definecolor{mygray}{rgb}{0.5,0.5,0.5}
\definecolor{mymauve}{rgb}{0.58,0,0.82}

\lstset{ 
  backgroundcolor=\color{white},   % choose the background color; you must add \usepackage{color} or \usepackage{xcolor}; should come as last argument
  basicstyle=\footnotesize,        % the size of the fonts that are used for the code
  breakatwhitespace=false,         % sets if automatic breaks should only happen at whitespace
  breaklines=true,                 % sets automatic line breaking
  captionpos=b,                    % sets the caption-position to bottom
  commentstyle=\color{mygreen},    % comment style
  deletekeywords={...},            % if you want to delete keywords from the given language
  escapeinside={\%*}{*)},          % if you want to add LaTeX within your code
  extendedchars=true,              % lets you use non-ASCII characters; for 8-bits encodings only, does not work with UTF-8
  firstnumber=1000,                % start line enumeration with line 1000
  frame=single,	                   % adds a frame around the code
  keepspaces=true,                 % keeps spaces in text, useful for keeping indentation of code (possibly needs columns=flexible)
  keywordstyle=\color{blue},       % keyword style
  language=Python,                 % the language of the code
  morekeywords={*,...},            % if you want to add more keywords to the set
  numbers=left,                    % where to put the line-numbers; possible values are (none, left, right)
  numbersep=5pt,                   % how far the line-numbers are from the code
  numberstyle=\tiny\color{mygray}, % the style that is used for the line-numbers
  rulecolor=\color{black},         % if not set, the frame-color may be changed on line-breaks within not-black text (e.g. comments (green here))
  showspaces=false,                % show spaces everywhere adding particular underscores; it overrides 'showstringspaces'
  showstringspaces=false,          % underline spaces within strings only
  showtabs=false,                  % show tabs within strings adding particular underscores
  stepnumber=2,                    % the step between two line-numbers. If it's 1, each line will be numbered
  stringstyle=\color{mymauve},     % string literal style
  tabsize=2,	                   % sets default tabsize to 2 spaces
  title=\lstname                   % show the filename of files included with \lstinputlisting; also try caption instead of title
}

\begin{document}

\title{Programski jezik Objective-C\\ \small{Seminarski rad u okviru kursa\\Metodologija stručnog i naučnog rada\\ Matematički fakultet}}

\author{Dalma Beara, Denis Aličić, Mateja Marjanović, Anja Miletić\\ kontakt email prvog, drugog, trećeg, četvrtog autora}

%\date{9.~april 2015.}

\maketitle

\abstract{
U ovom tekstu je ukratko prikazana osnovna forma seminarskog rada. Obratite pažnju da je pored ove .pdf datoteke, u prilogu i odgovarajuća .tex datoteka, kao i .bib datoteka korišćena za generisanje literature. Na prvoj strani seminarskog rada su naslov, apstrakt i sadržaj, i to sve mora da stane na prvu stranu! Kako bi Vaš seminarski zadovoljio standarde i očekivanja, koristite uputstva i materijale sa predavanja na temu pisanja seminarskih radova. Ovo je samo šablon koji se odnosi na fizički izgled seminarskog rada (šablon koji \emph{morate} da ispoštujete!) kao i par tehničkih pomoćnih uputstava. Pročitajte tekst pažljivo jer on sadrži i važne informacije vezane za zahteve obima i karakteristika seminarskog rada.}

\tableofcontents

\newpage

\section{Uvod}
\label{sec:uvod}

Kada budete predavali seminarski rad, imenujete datoteke tako da sadrže redni broj teme, temu seminarskog rada, kao i prezimena članova grupe. Precizna uputstva na temu imenovnja će biti data na formi za predaju seminarskog rada. Predaja seminarskih radova biće isključivo preko veb forme, a NE slanjem mejla. Link na formu će biti dat u okviru obaveštenja na strani kursa. Vodite računa da prilikom predavanja seminarskog rada predate samo one fajlove koji su neophodni za ponovno generisanje pdf datoteke. To znači da pomoćne fajlove, kao što su .log, .out, .blg, .toc, .aux i slično, \textbf{ne treba predavati}.

\section{Osnovna uputstva}
Vaš seminarski rad mora da sadrži najmanje jednu \textbf{sliku}, najmanje jednu \textbf{tabelu} i najmanje \textbf{sedam referenci} u spisku literature. Najmanje jedna slika treba da bude originalna i da predstavlja neke podatke koje ste Vi osmislili da treba da prezentujete u svom radu. Isto važi i za najmanje jednu tabelu. 	Od referenci, neophodno je imati bar jednu \textbf{knjigu}, bar jedan \textbf{naučni članak} iz odgovarajućeg časopisa i bar jednu adekvatnu \textbf{veb adresu}. 

\textbf{Dužina seminarskog rada treba da bude od 10 do 12 strana.}

Ко жели, може да пише рад ћирилицом. У том случају, неопходно је да су инсталирани одговарајући пакети: texlive-fonts-extra, texlive-latex-extra, texlive-lang-cyrillic, texlive-lang-other. 

Nemojte koristiti stari način pisanja slova, tj ovo:
\begin{verbatim}
\v{s} i \v{c} i \'c ...
\end{verbatim}
Koristite direknto naša slova:	
\begin{verbatim}
š i č i ć ... 
\end{verbatim}


\section{Engleski termini i citiranje}	
\label{sec:termini_i_citiranje}

Na svakom mestu u tekstu naglasiti odakle tačno potiču informacije. Uz sve novouvedene termine u zagradi naglasiti od koje engleske reči termin potiče. 

Naredni primeri ilustruju način uvođenja enlegskih termina kao i citiranje.

\begin{primer}
Problem zaustavljanja (eng.~{\em halting problem}) je neodlučiv \cite{haltingproblem}.
\end{primer}

\begin{primer}
Za prevođenje programa napisanih u programskom jeziku C može se koristiti GCC kompajler \cite{gcc}.
\end{primer}

\begin{primer}
 Da bi se ispitivala ispravost softvera, najpre je potrebno precizno definisati njegovo ponašanje \cite{laski2009software}. 
\end{primer}

Reference koje se koriste u ovom tekstu zadate su u datoteci {\em seminarski.bib}. Prevođenje u pdf format u Linux okruženju može se uraditi na sledeći način:
\begin{verbatim}
pdflatex TemaImePrezime.tex 
bibtex TemaImePrezime.aux 
pdflatex TemaImePrezime.tex 
pdflatex TemaImePrezime.tex 
\end{verbatim}
Prvo latexovanje je neophodno da bi se generisao {\em .aux} fajl. {\em bibtex} proizvodi odgovarajući {\em .bbl} fajl koji se koristi za generisanje literature. 
Potrebna su dva prolaza (dva puta pdflatex) da bi se reference ubacile u tekst (tj da ne bi ostali znakovi pitanja umesto referenci). Dodavanjem novih referenci potrebno je ponoviti ceo postupak.  











Broj naslova i podnaslova je proizvoljan. Neophodni su samo Uvod i Zaključak. Na poglavlja unutar teksta referisati se po potrebi. 
\begin{primer}
U odeljku \ref{sec:naslov1} precizirani su osnovni pojmovi, dok su zaključci dati u odeljku \ref{sec:zakljucak}.
\end{primer}

Još jednom da napomenem da nema razloga da pišete:
\begin{verbatim}
\v{s} i \v{c} i \'c ...
\end{verbatim}
Možete koristiti srpska slova
\begin{verbatim}
š i č i ć ... 
\end{verbatim}



\section{Slike i tabele}
\label{slike_i_tabele}

Slike i tabele treba da budu u svom okruženju, sa odgovarajućim naslovima, obeležene labelom da koje omogućava referenciranje. 

\begin{primer} Ovako se ubacuje slika. Obratiti pažnju da je dodato i 
\begin{verbatim}
\usepackage{graphicx}
\end{verbatim}

\begin{figure}[h!]
\begin{center}
\end{center}
\caption{Pande}
\label{fig:pande}
\end{figure}

Na svaku sliku neophodno je referisati se negde u tekstu. Na primer, na slici \ref{fig:pande} prikazane su pande. 
\end{primer}

\begin{primer} I tabele treba da budu u svom okruženju, i na njih je neophodno referisati se u tekstu. Na primer, u tabeli \ref{tab:tabela1} su prikazana različita poravnanja u tabelama.

\begin{table}[h!]
\begin{center}
\caption{Razlčita poravnanja u okviru iste tabele ne treba koristiti jer su nepregledna.}
\begin{tabular}{|c|l|r|} \hline
centralno poravnanje& levo poravnanje& desno poravnanje\\ \hline
a &b&c\\ \hline
d &e&f\\ \hline
\end{tabular}
\label{tab:tabela1}
\end{center}
\end{table}

\end{primer}

\section{K\^{o}d i paket listings}
Za ubacivanje koda koristite paket \textbf{listings}:
\url{https://en.wikibooks.org/wiki/LaTeX/Source_Code_Listings}

Primer ubacivanja koda
\begin{lstlisting}[frame=single]
# This program adds up integers in the command line
import sys
try:
    total = sum(int(arg) for arg in sys.argv[1:])
    print 'sum =', total
except ValueError:
    print 'Please supply integer arguments'
\end{lstlisting}


\section{Nastanak i istorijski razvoj, mesto u ravojnom stablu, uticaji drugih programskih jezika}
\label{sec:osnovno}

\subsection{Nastanak i istorijski razvoj}
\label{subsec:istorija}
Iako se pojam jezika Objective-C primarno vezuje za proizvode kompanije Epl (engl. Apple) -- MAC OS X, iPhone itd, on je zapravo nastao mnogo pre njih. Njegove temelje postavili su Bred Koks i Tom Lav 1981. godine, težeći da nađu način za povećavanje produktivnosti programera. Te godine pojavio se novi, revolucionarni jezik Smalltalk, koji je tada unapredio koncept objektno-orijentisanog programiranja. U osnovi tog jezika bilo je posmatranje programa kao skupova objekata koji su mogli da komuniciraju jedni sa drugima dinamički pozivajući metode. To je omogućilo da se stanje programa menja pod uticajem korisnika. Koks je u ovoj idej video mogućnost da se vrtoglavo ubrza pisanje programa, pošto su se u njemu mogle praviti biblioteke objekata i posle koristiti u drugim programima bez izmene. Međutim, Smalltalk je bio veoma spor i zahtevao je da se svi programi pišu i pokreću u posebnom okruženju. Tada se javila potreba za spajanjem objektno-orijentisanih ideja Smalltalk-a i brzog jezika C. Koks je 1983. objavio naučni rad u kom je predstavio objektno-orijentisani prekompilator (engl. Object-Oriented Precompiler). Da bi ovu svoju ideju izbacili na tržište, njih dvojica su osnovali kompaniju Stepstoun (engl. Stepstone), izmenili kompilator za OOPC i preimenovali jezik u Objective-C. Iako je glavna ideja ovog projekta bila pisanje i prodavanje pomenutih biblioteka koje su mogle ponovo da se koriste, korisnici su imali dosta zamerki, te je jezik sve više poprimao karakteristike objektno-orijentisanih jezika kakve ih danas znamo, kao što su sakupljač otpada i interpreter, te je kompanija uskoro propala. Stiv Džobs je 1988. kupio licencu, a nekoliko godina kasnije i sva prava za Objective-C za potrebe svoje kompanije NeXT Computer koja se  1996. pripojila Apple-u. Sa tom tranzicijom, sam jezik je pretrpeo razne promene, a najvažnije su uvođenje dveju novih ,,komponenti": kategorija (engl. categories), danas poznatih kao ekstenzije (engl. extensions) i protokola (engl. protocols). Kategorije su služile da programerima omoguće da sami dodaju funkcije u objekte iz već postojećih biblioteka, a protokoli su omogućavali objektima da komunuciraju jedni s drugima. Oni su danas podržani u jeziku Java i poznati su nam kao interfejsi (engl. interfaces). 

\subsection{Mesto u razvojnom stablu}
\label{subsec:stablo}


\subsection{Uticaj drugih programskih jezika}
\label{subsec:uticaj}
Kao što je već navedeno, Objective-C je nastao kao kombinacija koncepata na kojima počiva Smalltalk utočenih u sintaksu jezika C. Isprva je čak funkcionisao tako što je kod prevođen na C i onda izvršavan. S druge strane, on je uticao na jezik Swift koji je razvio Epl. Swift se često naziva i ,,Objective-C bez C-a". 

\section{Osnovna namena programskog jezika, svrha i mogućnosti}
\label{sec:namena}

\section{Osnovne osobine ovog programskog jezika, podržane paradigme i koncepti}
\label{sec:osobine}

\section{Najpoznatija okruženja (framework) za korišćenje ovog jezika i njihove karakteristike}
\label{sec:okruzenja}

\section{Instalacija i uputstvo za pokretanje na Linux/Windows operativnim sistemima}
\label{sec:instalacija}
\subsection{Instalacija neophodnih paketa}
Da bismo mogli da kompajliramo programe napisane u Objective-C, potrebno je prvo da instaliramo
par paketa. Za kompajliranje Objective-C programa koristi se gcc kompajler, uz jos par
dodataka.
Prvo sto treba da uradimo je da instaliramo gcc podrsku za Objective-C.

\begin{lstlisting}[frame=single]
$ sudo apt-get install gobjc
\end{lstlisting}

Zatim treba da instaliramo framework na kom se moderni Objective-C zasniva i bez kog
ne bi imalo puno smisla raditi (iako je moguce).
Taj framework se zove GNUstep i instalira se komandom

\begin{lstlisting}[frame=single]
$ sudo apt-get install gnustep
$ sudo apt-get install gnustep-devel
\end{lstlisting}

\section{Primer jednostavnog koda i njegovo objašnjenje}
\label{sec:primer}

Zatim, navedimo primer zastarelog (od verzije MAC 10.5), ali zanimljivog koda koji omogućava ,,imitiranje" (engl. posing). 

\begin{lstlisting}[frame=single]
#import <Foundation/Foundation.h>

@interface MyString : NSString

@end

@implementation MyString

- (NSString *)stringByReplacingOccurrencesOfString:(NSString *)target
withString:(NSString *)replacement {
   NSLog(@"The Target string is %@",target);
   NSLog(@"The Replacement string is %@",replacement);
}

@end

int main() {
   NSAutoreleasePool * pool = [[NSAutoreleasePool alloc] init];
   [MyString poseAsClass:[NSString class]];
   NSString *string = @"Test";
   [string stringByReplacingOccurrencesOfString:@"a" withString:@"c"];
   
   [pool drain];
   return 0;
}
\end{lstlisting}

\section{Sve ono što je specifično i važno za sam taj programski jezik}
\label{sec:naslovN}

Ovde pišem tekst. 
Ovde pišem tekst. 
Ovde pišem tekst. 
Ovde pišem tekst. 
Ovde pišem tekst. 

\subsection{... podnaslov}
\label{subsec:podnaslovK}

Ovde pišem tekst. 
Ovde pišem tekst. 
Ovde pišem tekst. 
Ovde pišem tekst. 
Ovde pišem tekst. 

\subsection{... podnaslov}
\label{subsec:podnaslovM}

Ovde pišem tekst. 
Ovde pišem tekst. 
Ovde pišem tekst. 
Ovde pišem tekst. 
Ovde pišem tekst. 


\section{Zaključak}
\label{sec:zakljucak}

Ovde pišem zaključak. 
Ovde pišem zaključak. 
Ovde pišem zaključak. 
Ovde pišem zaključak. 
Ovde pišem zaključak. 
Ovde pišem zaključak. 
Ovde pišem zaključak. 
Ovde pišem zaključak. 
Ovde pišem zaključak. 
Ovde pišem zaključak. 
Ovde pišem zaključak. 
Ovde pišem zaključak. 


\addcontentsline{toc}{section}{Literatura}
\appendix
\bibliography{seminarski} 
\bibliographystyle{plain}

\appendix
\section{Dodatak}
Ovde pišem dodatne stvari, ukoliko za time ima potrebe.
Ovde pišem dodatne stvari, ukoliko za time ima potrebe.
Ovde pišem dodatne stvari, ukoliko za time ima potrebe.
Ovde pišem dodatne stvari, ukoliko za time ima potrebe.
Ovde pišem dodatne stvari, ukoliko za time ima potrebe.


\end{document}