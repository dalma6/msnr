% !TEX encoding = UTF-8 Unicode

\documentclass[a4paper]{report}

\usepackage[T2A]{fontenc} % enable Cyrillic fonts
\usepackage[utf8x,utf8]{inputenc} % make weird characters work
\usepackage[serbian]{babel}
%\usepackage[english,serbianc]{babel}
\usepackage{amssymb}

\usepackage{color}
\usepackage{url}
\usepackage[unicode]{hyperref}
\hypersetup{colorlinks,citecolor=green,filecolor=green,linkcolor=blue,urlcolor=blue}

\newcommand{\odgovor}[1]{\textcolor{blue}{#1}}

\begin{document}

\title{ObjectiveC\\ \small{Dopunite autore rada}}

\maketitle

\tableofcontents

\chapter{Uputstva}
\emph{Prilikom predavanja odgovora na recenziju, obrišite ovo poglavlje.}

Neophodno je odgovoriti na sve zamerke koje su navedene u okviru recenzija. Svaki odgovor pišete u okviru okruženja \verb"\odgovor", \odgovor{kako bi vaši odgovori bili lakše uočljivi.} 
\begin{enumerate}

\item Odgovor treba da sadrži na koji način ste izmenili rad da bi adresirali problem koji je recenzent naveo. Na primer, to može biti neka dodata rečenica ili dodat pasus. Ukoliko je u pitanju kraći tekst onda ga možete navesti direktno u ovom dokumentu, ukoliko je u pitanju duži tekst, onda navedete samo na kojoj strani i gde tačno se taj novi tekst nalazi. Ukoliko je izmenjeno ime nekog poglavlja, navedite na koji način je izmenjeno, i slično, u zavisnosti od izmena koje ste napravili. 

\item Ukoliko ništa niste izmenili povodom neke zamerke, detaljno obrazložite zašto zahtev recenzenta nije uvažen.

\item Ukoliko ste napravili i neke izmene koje recenzenti nisu tražili, njih navedite u poslednjem poglavlju tj u poglavlju Dodatne izmene.
\end{enumerate}

Za svakog recenzenta dodajte ocenu od 1 do 5 koja označava koliko vam je recenzija bila korisna, odnosno koliko vam je pomogla da unapredite rad. Ocena 1 označava da vam recenzija nije bila korisna, ocena 5 označava da vam je recenzija bila veoma korisna. 

NAPOMENA: Recenzije ce biti ocenjene nezavisno od vaših ocena. Na osnovu recenzije ja znam da li je ona korisna ili ne, pa na taj način vama idu negativni poeni ukoliko kažete da je korisno nešto što nije korisno. Vašim kolegama šteti da kažete da im je recenzija korisna jer će misliti da su je dobro uradili, iako to zapravo nisu. Isto važi i na drugu stranu, tj nemojte reći da nije korisno ono što jeste korisno. Prema tome, trudite se da budete objektivni. 
\chapter{Recenzent \odgovor{--- ocena:} }


\section{O čemu rad govori?}
U prvom delu rada je detaljno prikazan razvoj programskog jezika Objective-C kroz istoriju kao i uticaj njegovih prethodnika i korisnika na trenutni izgled jezika. Nisu izostavljeni ni jezici nalsednici koji su na kraju ovog dela rada uključeni u razvojno stablo programskog jezika.

Prikazano je u kojim aplikacijama i operativnim sistemima se koristi Objective-C. Navedeni su i opisani koncepti objektno-orijentisanog programiranja koje Objective-C podržava(klasteri klasa, introspekcija, alokacija objekata i MVC) bez ulaženja u detalje konkretne implementacije u jeziku osim u slučaju introspekcije gde je dat primer koda.

Dalje u radu su redom opisana razvojna okruzenja: Cocoa koji se koristi na Apple Mac OS operativnim sistemima za razvoj aplikacija(Cocoa Touch dodatak za razvijanje aplikacija na mobilnim uredjajima) kao i istorijat Cocoa API-ja; GNUStep koji se koristi za razvoj aplikacija na Linux, Solaris, BSD i Windows operativnim sistemima; ostali korisni alati i integrisana razvojna okruženja za razvoj aplikacija.

Dato je detaljno upustvo instalacije neophodnih paketa za kompajliranje Objective-C aplikacija na Linux sistemima pomoću gcc kompajlera, instalacije razvojnog okruženja, kompajliranja i pokretanja aplikacije. Takođe je objašnjen proces razvoja aplikacija na Windows sistemima.

Na samom kraju rada dat je primer jednostavnog koda u programskom jeziku Objective-C zajedno sa objašnjenjima delova koda i opisom korišćene sintakse jezika.

\section{Krupne primedbe i sugestije}
% Напишете своја запажања и конструктивне идеје шта у раду недостаје и шта би требало да се промени-измени-дода-одузме да би рад био квалитетнији.
Druga sekcija o istorijatu je preopširna za rad ovog obima i moglo bi se uštedeti na prostoru u ovoj sekciji, a dodati još detalja u sekciji o osnovnim osobinama i konceptima programskog jezika koja bi trebala biti najznačajnija u radu(primeri koda, detaljnija objašnjenja o implementaciji).

Treća sekcija o nameni i mogućnostima programskog jezika zbog svog malog obima može biti uključena kao podsekcija četvrte sekcije.

Četvta sekcija predstavlja temu od najvećeg značaja za rad i trebalo bi da bude najopširnija i najkvalitetnije obrađena.

U petoj sekciji se može izostaviti deo o istorijatu Cocoa API-ja zbog ne toliko velike relevantnosti za rad kao i prevelikog obima. Još jedna opcija bi bila sažimanje ovog dela i spajanje sa delom o Cocoa razvojnom okruženju. \odgovor{Mislim da je Cocoa istorijat bitan zbog razumevanja naziva tipova podataka u samom API-u, bitnost za sam programski jezik, kao i da 8 rečenica nije preobimno.}

Sedma sekcija sa primerom koda bi mogla biti proširena sa još nekoliko primera kako bi čitalac nakon pročitanog rada mogao imati predstavu kako konstruisati osnovni program u programskom jeziku Objective-C.

U radu je bilo akcenta na detaljima koji nisu od suštinskog značaja za ovaj obim rada, ali ukoliko bi se rad proširio za nekoliko stranica ne bi bilo potrebno sažimanje ostalih sekcija kao što je prethodno navedeno već bi bilo dovoljno dodati par sekcija koje su usko vezane za programiranje u zadatom jeziku. Proširivanjem rad bi bio kompletan i uz eventualno minimalno sažimanje(previše istoriografije) ne bi bilo krupnijih zamerki na rad.



\section{Sitne primedbe}
\begin{enumerate}
	
	\item Podsekcija 2.2, prvi pasus, poslednja rečenica: \\
	\textquotedbl Swift se često naziva i 'Objective-C bez C-a' kao i na Javu.\textquotedbl{}. Deo \textquotedbl kao i na Javu\textquotedbl{} nema smisla u navedenoj rečenici pa je najverovatnije deo prethodne rečenice \textquotedbl S druge
	strane, on je uticao na jezik Swift koji je razvio Epl.\textquotedbl{}.
	
	\item Podsekcija 4.1.1, prvi pasus, druga rečenica: \\
	\textquotedbl Klasteri klasa grupišu
	odredjeni broj privatnih podklasa unutar javne apstraktne nadklase.\textquotedbl{}. Slovo 'đ' u reči 'odredjeni' nije napisano srpskom latinicom.
	
	\item Podsekcija 4.1.2, prvi pasus, druga rečenica: \\
	\textquotedbl Ti detalji mogu biti njegovo
	mesto u drvetu nasledjivanja, da li implementira odredjeni protokol i da li odgovara na neku poruku.\textquotedbl{}. Slovo 'đ' u rečima 'odredjeni' i 'nasledjivanja' nije napisano srpskom latinicom.
	
	\item Podsekcija 5.1.1, drugi pasus, druga rečenica: \\
	\textquotedbl Originalno je bila poznata kao KidSim, a sada je licencirana drugoj firmi kao Stagecast Creator.\textquotedbl{}. Ime programera 'Kid Sim' je napisano sastavljeno i izostavljeno je 'u' u delu 'licencirana drugoj firmi'.
	\odgovor{,,KidSim'' nije ime programera, već prvobitno ime Cocoa projekta.}
	
\end{enumerate}

\section{Provera sadržajnosti i forme seminarskog rada}
% Oдговорите на следећа питања --- уз сваки одговор дати и образложење

\begin{enumerate}
\item Da li rad dobro odgovara na zadatu temu?\\
Rad uglavnom odgovara na zadatu temu, ali je u nekim delovima akcenat na stvarima koje nisu od suštinskog značaja za temu.

\item Da li je nešto važno propušteno?\\
Propušteno je da se detaljnije obrade podržane paradigme i koncepti jezika kao i osnovna implementacija konkretnih paradigmi i koncepata.

\item Da li ima suštinskih grešaka i propusta?\\
Rad nema suštinskih grešaka i propusta.

\item Da li je naslov rada dobro izabran?\\
Naslov rada je  dobro izabran i iz njega se može zaključiti o temi obrađenoj unutar seminarskog rada.

\item Da li sažetak sadrži prave podatke o radu?\\
Sažetak sadrži odgavarajuće podatke o radu čime nam daje predstavu o temama obrađenim unutar seminarskog rada.

\item Da li je rad lak-težak za čitanje?\\
Rad je pretežno lak za čitanje.

\item Da li je za razumevanje teksta potrebno predznanje i u kolikoj meri?\\
Za većinu tema obrađenih u radu nije potrebno predznanje za razumevanje, dok je za razumevanje sekcije o konceptima jezika potrebno osnovno predznanje o objektno-orijentisanom programiranju.
 
\item Da li je u radu navedena odgovarajuća literatura?\\
Literatura korišćena u radu je navedena i relevantna je za obrađenu temu. Literatura je iz pouzdanih izvora.

\item Da li su u radu reference korektno navedene?\\
Sve reference unutar rada su korektno navedene.

\item Da li je struktura rada adekvatna?\\
Struktura rada je adekvatna.

\item Da li rad sadrži sve elemente propisane uslovom seminarskog rada (slike, tabele, broj strana...)?\\
Rad ne sadrži propisani broj strana(sadrži 8 strana dok je zahtevano između 10 i 12 strana), dok su svi ostali zahtevani elementi ispunjeni.

\item Da li su slike i tabele funkcionalne i adekvatne?\\
Slike i tabele unutar rada su adekvatne za obrađenu temu i slikovitim objašnjavanjem odnosa između programskih jezika radi lakšeg razumevanja dokazuju da su funkcionalne.
\end{enumerate}

\section{Ocenite sebe}
% Napišite koliko ste upućeni u oblast koju recenzirate: 
% a) ekspert u datoj oblasti
% b) veoma upućeni u oblast
% c) srednje upućeni
% d) malo upućeni 
% e) skoro neupućeni
% f) potpuno neupućeni
% Obrazložite svoju odluku
Za sebe smatram da sam srednje upućen u oblast koju recenziram jer konkretno o programskom jeziku Objective-C nemam adekvatno predznanje.

\chapter{Recenzent \odgovor{--- ocena:2} }


\section{O čemu rad govori?}
% Напишете један кратак пасус у којим ћете својим речима препричати суштину рада (и тиме показати да сте рад пажљиво прочитали и разумели). Обим од 200 до 400 карактера.
Ovaj rad govori o programskom jeziku Objectiv-C, koncentrišući se prvenstveno na njegov nastanak, razvoj i uticaj na druge jezike, a zatim i na specifične upotrebe. Takođe je korektno opisan i način instalacije i upotrebe u poznatim razvojnim okruženjima. Pomenuti su i uvedeni opisani pojmovi klasteri klasa, introspekcija i alokacija objekata u ovom jeziku.
\section{Krupne primedbe i sugestije}
% Напишете своја запажања и конструктивне идеје шта у раду недостаје и шта би требало да се промени-измени-дода-одузме да би рад био квалитетнији.
U tekstu se na dosta mesta ponavlja informacija o uticaju jezika na druge jezike(poglavlje 2.1, 2.2 i 3) i naglašava njegova objektno-orjentisana paradigma(poglavlje 1, 2.1, 2.2, 3 ). Ispod pojedinih naslova nedostaje pasus, nego je prvi tekst ispod novi podnaslov(poglavlje 2, 5, 6).  Kodovi su navedi bez pozivanja na odgovarajući listing. U poglavlju sa kratkim pitanjima u ovoj recenziji skrenuta je pažnja na još neke nedostatke.
\section{Sitne primedbe}
% Напишете своја запажања на тему штампарских-стилских-језичких грешки
Skraćenica OOPC je uvedena bez prethodnog objašnjenja(poglavlje 2.1). U nekim rečima slovo đ je pisano kao dj(nasledjuje, takodje)(poglavlje 3, 4, 4.1.1, 4.1.2).Sedmi pasus počinje sa 'zaista' što nije preporučeni način.
\section{Provera sadržajnosti i forme seminarskog rada}
% Oдговорите на следећа питања --- уз сваки одговор дати и образложење

\begin{enumerate}
\item Da li rad dobro odgovara na zadatu temu?\\
Rad odgovara temi, ali u njemu nisu na istom nivou opisani svi potrebni i osnovni koncepti programskog jezika.
\item Da li je nešto važno propušteno?\\
Propuštena je osnovna priča o sintaksi i osnovnim pojmovima(funkcije, tipovi i sl.) i primeri koda koji su dati predstavljaju napredne koncepte koji nisu razumljivi bez opisa jednostanijih konstrukcija jezika.
\item Da li ima suštinskih grešaka i propusta?\\
Nema suštinskih grešaka. 
\item Da li je naslov rada dobro izabran?\\
Naslov rada je klasičan i opšti. Mogao bi da sadrži nešto što ga bolje opisuje ili nešto što bi privuklo čitaoca.
\item Da li sažetak sadrži prave podatke o radu?\\
Sažetak je korektan. Razmotriti dodavanje ciljeva rada.
\item Da li je rad lak-težak za čitanje?\\
Zbog nedostatka opšteg znanja o jeziku, deo gde se opisuju specifičnosti je potrebno čitati sa više pažnje i uz dalje samostalno istrživanje.
\item Da li je za razumevanje teksta potrebno predznanje i u kolikoj meri?\\
Potrebno je predzanje iz osnova jezika kako bi se razumele komplikovanije stvari.
\item Da li je u radu navedena odgovarajuća literatura?\\
U literaturi fali naučni časopis. Vidno dominantna literatura internet sajtovi.
\item Da li su u radu reference korektno navedene?\\
Neki delovi teksta nisu pokrivene referencama na literaturu. 
\item Da li je struktura rada adekvatna?\\
Neke teme(podnaslovi) su izraženije više istražene od drugih. Poglavlje 3 previše kratko da bi se našlo kao posebna celina i u njemu se ponavlja dosta već rečenog ranije. MVC arhitektura je opisana bez jasno definisane veze sa ovim programskim jezikom i njegove uloge u istoj.
\item Da li rad sadrži sve elemente propisane uslovom seminarskog rada (slike, tabele, broj strana...)?\\
Rad ne sadrži propisani minimum od 10 strana. Ostali elementi su prisutni.
\item Da li su slike i tabele funkcionalne i adekvatne?\\
Informacije u tabeli su nešto što je već obrazloženo u tekstu. Preporučuje se i generisanje bar jednog grafika.Predlog je da reference na slike budu u obliku koji se uklapa u tekst ("Na slici 1..."), jer samo navođenje broja slike u tekstu na kraju rečenice može biti zbunjujuće.
\end{enumerate}

\section{Ocenite sebe}
% Napišite koliko ste upućeni u oblast koju recenzirate: 
% a) ekspert u datoj oblasti
% b) veoma upućeni u oblast
% c) srednje upućeni
% d) malo upućeni 
% e) skoro neupućeni
% f) potpuno neupućeni
% Obrazložite svoju odluku
Nisam se do sada susrela sa ovim programskim jezikom tako da sam skoro neupućena. Nakon čitanja rada imam bolju sliku o njemu, ali i dalje nedovoljno primenljivog znanja.


\chapter{Dodatne izmene}
%Ovde navedite ukoliko ima izmena koje ste uradili a koje vam recenzenti nisu tražili. 

\end{document}
